\begin{itemize}
\tightlist
\item
  To start the simulation use the function \texttt{sim} and specify the
  number of runs.
\item
  To start the simulation in parallel define the \texttt{mode} and
  number of \texttt{cpus}.
\item
  Packages and also objects can be loaded on all nodes with extra
  arguments to \texttt{sim}.
\item
  Here we perform 16 runs on 4 cores. \texttt{sim} always returns a
  \texttt{list} and we combine the results directly using
  \texttt{rbind\_all}.
\end{itemize}

\begin{Shaded}
\begin{Highlighting}[]
\KeywordTok{sim}\NormalTok{(setup, }\DataTypeTok{R =} \DecValTok{16}\NormalTok{, }\DataTypeTok{mode =} \StringTok{"socket"}\NormalTok{, }\DataTypeTok{cpus =} \DecValTok{4}\NormalTok{) %>%}\StringTok{  }\NormalTok{dplyr::}\KeywordTok{bind_rows}\NormalTok{() %>%}\StringTok{ }\KeywordTok{dim}\NormalTok{()}
\end{Highlighting}
\end{Shaded}

\begin{verbatim}
## Starting parallelization in mode=socket with cpus=4.
\end{verbatim}

\begin{verbatim}
## Mapping in parallel: mode = socket; cpus = 4; elements = 16.
\end{verbatim}

\begin{verbatim}
## Stopped parallelization. All cleaned up.
\end{verbatim}

\begin{verbatim}
## [1] 160000      8
\end{verbatim}
