\begin{itemize}
\itemsep1pt\parskip0pt\parsep0pt
\item
  Set the frequency or probability for adding contaminated observations
\item
  Specify contamination within domains or across the population
\item
  Change between area and unit level contamination
\end{itemize}

\begin{Shaded}
\begin{Highlighting}[]
\NormalTok{setup %>%}\StringTok{ }
\StringTok{  }\KeywordTok{sim_gen_cont}\NormalTok{(}\KeywordTok{gen_generic}\NormalTok{(rnorm, }\DataTypeTok{sd =} \DecValTok{150}\NormalTok{, }\DataTypeTok{name =} \StringTok{"e"}\NormalTok{), }\FloatTok{0.1}\NormalTok{, }\StringTok{"unit"}\NormalTok{)}
\end{Highlighting}
\end{Shaded}

\begin{verbatim}
## data.frame [10,000 x 7]
## 
##    idD idU         x         e        v   idC         y
## 1    1   1 -2.166447 -1.029188 1.208294 FALSE  95.84621
## 2    1   2 -4.770198  2.172307 1.208294 FALSE  93.84021
## 3    1   3  2.393438 -4.610486 1.208294 FALSE 101.38468
## 4    1   4  2.754315 -5.162324 1.208294 FALSE 101.55460
## 5    1   5  2.735989 -3.642082 1.208294 FALSE 103.03819
## 6    1   6 -2.055526  5.921647 1.208294 FALSE 103.01889
## .. ... ...       ...       ...      ...   ...       ...
\end{verbatim}

\begin{Shaded}
\begin{Highlighting}[]
\KeywordTok{autoplot}\NormalTok{(setup)}
\KeywordTok{autoplot}\NormalTok{(setup %>%}\StringTok{ }\KeywordTok{sim_gen_vc}\NormalTok{()) }\CommentTok{# contamination on area-level }
\end{Highlighting}
\end{Shaded}

