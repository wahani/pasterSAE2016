\begin{Shaded}
\begin{Highlighting}[]
\NormalTok{setup1 <-}\StringTok{ }\KeywordTok{sim_base_lm}\NormalTok{() %>%}\StringTok{ }\KeywordTok{sim_sample}\NormalTok{(}\KeywordTok{sample_number}\NormalTok{(}\DecValTok{5}\NormalTok{))}
\NormalTok{setup2 <-}\StringTok{ }\KeywordTok{sim_base_lm}\NormalTok{() %>%}\StringTok{ }\KeywordTok{sim_sample}\NormalTok{(}\KeywordTok{sample_fraction}\NormalTok{(}\FloatTok{0.05}\NormalTok{))}
\end{Highlighting}
\end{Shaded}

\begin{itemize}
\itemsep1pt\parskip0pt\parsep0pt
\item
  \texttt{setup1} and \texttt{setup2} differ in the specific way samples
  are drawn. \texttt{sim\_sample} is responsible to find the position in
  the process
\item
  Every \texttt{sim\_*} function expects a simulation setup or
  \texttt{data.frame} as first argument
\item
  Every \texttt{sim\_*} controls at which position in the process a
  function is called
\item
  For every step in the process tools are named using the corresponding
  prefix, i.e. \texttt{gen\_generic} or \texttt{sample\_fraction}
\end{itemize}
