\begin{Shaded}
\begin{Highlighting}[]
\NormalTok{setup1 <-}\StringTok{ }\NormalTok{setup %>%}\StringTok{ }\KeywordTok{sim_sample}\NormalTok{(}\KeywordTok{sample_number}\NormalTok{(}\DecValTok{5}\NormalTok{))}
\NormalTok{setup2 <-}\StringTok{ }\NormalTok{setup %>%}\StringTok{ }\KeywordTok{sim_sample}\NormalTok{(}\KeywordTok{sample_fraction}\NormalTok{(}\FloatTok{0.05}\NormalTok{))}
\end{Highlighting}
\end{Shaded}

\begin{itemize}
\itemsep1pt\parskip0pt\parsep0pt
\item
  \texttt{setup1} and \texttt{setup2} only differ in the specific way
  samples are drawn.
\item
  \texttt{sim\_sample} inserts the sampling function at the appropriate
  position in the process.
\item
  For every step in the process tools are named using the corresponding
  prefix, e.g. \texttt{gen\_generic} for \textit{data generation} or
  \texttt{sample\_fraction} for \textit{drawing samples}.
\end{itemize}
