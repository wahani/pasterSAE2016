\begin{itemize}
\itemsep1pt\parskip0pt\parsep0pt
\item
  You can use predefined generators, e.g.~for spatially correlated
  variates, or include univariate random number generators from
  \texttt{R}.
\item
  \texttt{sim\_gen} defines the position in the chain.
  \texttt{gen\_generic} controls the generation process and accepts any
  random number generator as argument. Also we defined shortcuts like
  \texttt{sim\_gen\_generic} to be less verbose.
\item
  In the following you see a definition for drawing numbers from a
  linear mixed model. The response is constructed with an R expression
  and set by \texttt{sim\_resp\_eq}.
\end{itemize}

\begin{Shaded}
\begin{Highlighting}[]
\NormalTok{setup <-}\StringTok{ }\KeywordTok{sim_base}\NormalTok{() %>%}\StringTok{ }\CommentTok{# default with 100 areas (domains) and 100 units each}
\StringTok{  }\KeywordTok{sim_gen}\NormalTok{(}\KeywordTok{gen_generic}\NormalTok{(rnorm, }\DataTypeTok{sd =} \DecValTok{4}\NormalTok{, }\DataTypeTok{name =} \StringTok{"x"}\NormalTok{)) %>%}\StringTok{ }
\StringTok{  }\KeywordTok{sim_gen}\NormalTok{(}\KeywordTok{gen_generic}\NormalTok{(rnorm, }\DataTypeTok{sd =} \DecValTok{4}\NormalTok{, }\DataTypeTok{name =} \StringTok{"e"}\NormalTok{)) %>%}
\StringTok{  }\KeywordTok{sim_gen_generic}\NormalTok{(rnorm, }\DataTypeTok{sd =} \DecValTok{2}\NormalTok{, }\DataTypeTok{groupVars =} \StringTok{"idD"}\NormalTok{, }\DataTypeTok{name =} \StringTok{"v"}\NormalTok{) %>%}
\StringTok{  }\KeywordTok{sim_resp_eq}\NormalTok{(}\DataTypeTok{y =} \DecValTok{100} \NormalTok{+}\StringTok{ }\DecValTok{2} \NormalTok{*}\StringTok{ }\NormalTok{x +}\StringTok{ }\NormalTok{v +}\StringTok{ }\NormalTok{e)}
\NormalTok{setup}
\end{Highlighting}
\end{Shaded}

\begin{verbatim}
## data.frame [10,000 x 6]
## 
##    idD idU          x          e        v         y
## 1    1   1  0.7454883  3.6822042 1.459663 106.63284
## 2    1   2  5.6934259 -0.2506387 1.459663 112.59588
## 3    1   3  1.6492738 -1.2109953 1.459663 103.54722
## 4    1   4  2.4255113  2.5900518 1.459663 108.90074
## 5    1   5 -0.6598988 -5.8960966 1.459663  94.24377
## 6    1   6 -8.6498785 -1.6075882 1.459663  82.55232
## .. ... ...        ...        ...      ...       ...
\end{verbatim}
